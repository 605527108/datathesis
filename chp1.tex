\chapter{背景及研究现状}

\section{背景}
随着互联网的普及,人们每天花费越来越多时间通过网络工作,娱乐。目前,得益于芯片技术极速发展,比尔盖茨的目标,“地球上每个人都有个人电脑”,已基本实现,其中有40\%的电脑连接着互联网;全球大概30亿人持有计算能力相当于1980年房间般大小的超级计算机的智能手机,这些移动手机用户使用各地运营商大力推广的4G上网服务,实现随时随地上网的梦想。除了电脑和手机,人们还将可穿戴设备及各类物联网接入互联网中。因此,互联网各网络节点每天都能采集到大量用户的上网信息。基于这些用户信息,分析用户行为和研究用户模型,我们能挖掘出用户的兴趣爱好,由此推荐相关的内容,从而提高互联网的交互能力以及网络运营商内容运营能力。

在多篇论文中,研究人员已经对用户行为进行了细致的研究。比如使用K-means距离算法\parencite{杨清龙2013基于网络日志的互联网用户行为分析}把行为类似的用户分为几组;又如研究上下文推断用户连续性爱好\parencite{史艳翠2013基于通信数据的上下文移动用户偏好动态获取方法研究};也有使用支持向量机给用户类型分类的\parencite{程辉2013网络用户偏好分析及话题趋势预测方法研究};还有研究用户偏好随群体交流改变的\parencite{张欢2014网络用户偏好分析方法的研究,程辉2013网络用户偏好分析及话题趋势预测方法研究}。根据协同过滤算法\parencite{resnick1994grouplens},用户行为的建模可以描述为一个协同过滤中的User-Item矩阵。为了得到这个矩阵,本文要实现一个协同过滤的推荐系统。

目前,大量互联网公司采用推荐系统给用户推荐相关的内容。
全球最大网上购物平台Amazon最早将推荐系统引入网络销售领域\parencite{Linden2010Amazon}。
之后,推荐系统被应用于各种类型的互联网服务中。
比如影视服务方面,视频网站Netflix将推荐系统与A/B testing结合起来\parencite{Gomez2015The}。
Google旗下的YouTube给用户推荐视频\parencite{Davidson2010The}。
本文所介绍的系统与前者有所不同。
本文从互联网提供商(ISP)的流量分析入手。
由于大多数用户通过ISP上网,ISP的节点路由或者交换机能采集到用户访问各种类型的网站的流量。
这与之前提到的公司不同,他们只关注与公司业务相关的流量。
从ISP的角度看,我们可以全天候地提供全方位的推荐。

由Google带领的全球互联网进入HTTPS时代,许多互联网服务使用HTTPS,即将网页等网络信息进行TSL或者SSL加密传输。
这就意味着许多网络数据虽然流经ISP的路由或者交换机,但是ISP无法查看数据包中的内容。
但是目前作为领导者的Google承认自己有25\%的数据传输没有加密。
并且,日前以色列大学的研究团队最近发表一篇论文显示他们使用统计计算方法分析https加密流量,能够以96\%的准确性确定用户的操作系统、浏览器以及他们正在浏览的网页。
这就表明有大部分的网络数据是可以读取的。

本文就从学术的角度,探讨推荐系统实现的可行性及其中的关键技术。

\section{本文结构}
本文主要目标是研究用户模型,得出代表用户模型的User-Item矩阵,这些都基于本文提出的基于大数据分析的推荐系统。
其中第一章主要介绍了相关背景及研究现状。
第二章主要介绍相关算法及简单叙述Hadoop分布式数据处理的概念。
第三章主要介绍用户上网数据采集,网站语义分析,数据处理及存储,基于Hadoop的推荐算法等模块的工作细节。
第四章主要介绍整体系统工作流程,以及实证分析。
第五章总结以上内容。