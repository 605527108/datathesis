\documentclass{beamer}
\usepackage{ctex}
\usepackage{tikz}
\usepackage{pgfplots}
\usepackage{beamerthemesplit}
\title{基于互联网日志的用户行为分析算法和用户模型的研究}
\author{孙卓豪}
\institute{南开大学 \& 电子信息科学与技术}
\date{\today}

\begin{document}
\usetikzlibrary{shapes,snakes}
\usetikzlibrary{arrows, decorations.markings}
%%%%%%%%%%%%%%%%%%%%%%
\begin{frame}
	\titlepage
\end{frame}
%%%%%%%%%%%%%%%%%%%%%%

%%%%%%%%%%%%%%%%%%%%%%
\begin{frame}[shrink]
	\tableofcontents
\end{frame}
%%%%%%%%%%%%%%%%%%%%%%

\section{主要工作及相关内容解释}
\subsection{主要工作}
%%%%%%%%%%%%%%%%%%%%%%
\begin{frame}
\frametitle{主要工作及相关内容解释}
	\begin{block}{主要工作}
	本文的主要工作有:
	\begin{itemize}
		\item 使用Hadoop和MapReduce分析、处理大数据;
		\item 实现分布式协同过滤推荐系统;
		\item 通过用户兴趣矩阵构建用户模型。
	\end{itemize}
	\end{block}
\end{frame}
%%%%%%%%%%%%%%%%%%%%%%

\subsection{互联网日志的特点}
%%%%%%%%%%%%%%%%%%%%%%
\begin{frame}
	\frametitle{主要工作及相关内容解释}
	\begin{block}{互联网日志的特点}
	随着互联网的高速发展,网络用户留下越来越多的网络行为信息。

	互联网日志是保存于各种网络设备,如路由器、交换机或者服务器中用户使用网络服务的历史记录。
	互联网日志的特点:
	\begin{itemize}
		\item 数量巨大,分布范围广;
		\item 具有时效性;
		\item 隐含信息丰富。
	\end{itemize}
	从以上特点可以看出,我们可以从互联网日志中分析用户行为信息。而要处理大量的、分布的互联网日志,则需要使用Hadoop与MapReduce分布式计算平台。
	\end{block}
\end{frame}
%%%%%%%%%%%%%%%%%%%%%%

\subsection{用户爱好的模型及其表现形式}
%%%%%%%%%%%%%%%%%%%%%%
\begin{frame}
\frametitle{主要工作及相关内容解释}
	\begin{block}{用户爱好的模型及其表现形式}
	用户的某次网络行为通常与几个事件或者物品相关联,我们可以将这些事件或者物品转换为关键词的形式,然后构建以下User-Item矩阵,该矩阵记录用户与关键词的交互信息。
	\begin{center}
		\begin{tabular}{@{} c @{}|@{} c @{}|@{} c @{}|@{} c @{}|@{} c @{}|@{} c @{}}
			\hline
			 & Item 1 & Item 2 & Item 3 \\
			\hline
			User 1 & ? & 0.8 & 0.9 \\
			\hline
			User 2 & 0.1 & 0.6 & ? \\
			\hline
			User 3 & 0.2 & ? & 0.3 \\
			\hline
		\end{tabular}
		\end{center}
	\end{block}
\end{frame}
%%%%%%%%%%%%%%%%%%%%%%

\section{系统整体介绍}
%%%%%%%%%%%%%%%%%%%%%%
\begin{frame}
	\frametitle{系统整体介绍}
\tikzstyle{vecArrow} = [thick, decoration={markings,mark=at position
   1 with {\arrow[semithick]{open triangle 60}}},
   double distance=1.4pt, shorten >= 5.5pt,
   preaction = {decorate},
   postaction = {draw,line width=1.4pt, white,shorten >= 4.5pt}]

% Define box and box title style
\tikzstyle{mybox} = [draw=black, fill=white, very thick,
    rectangle, rounded corners, inner sep=10pt, inner ysep=20pt]
\tikzstyle{fancytitle} =[fill=red, text=white]

\def\html{
    \draw[rounded corners=0.1ex,fill=white,thick] (0.2,0.2) rectangle (0.9,1.2);
    \draw[rounded corners=0.1ex,fill=white,thick] (0.1,0.1) rectangle (0.8,1.1);
    \draw[rounded corners=0.1ex,fill=white,thick] (0,0) rectangle (0.7,1);
    \node at (0.35,0.2) {\tiny HTML};
}
\begin{center}
\begin{tikzpicture}
\node [mybox] (stepone){
	\begin{minipage}{0.40\textwidth}
		\tiny
        \begin{center}
		\begin{tabular}{@{} c @{}|@{} c @{}|@{} c @{}|@{} c @{}|@{} c @{}|@{} c @{}}
			\hline
			User ID & Request URL & Time & Location & Content & Device \\
			\hline
			1 & www.$\ldots$/bbb & 20:10 & hill & $\ldots$ & iPhone \\
			\hline
			2 & www.$\ldots$/aab & 23:20 & base & $\ldots$ & iPhone \\
			\hline
			3 & www.$\ldots$/cdd & 24:10 & grid & $\ldots$ & iPhone \\
			\hline
		\end{tabular}
		\end{center}
	\end{minipage}
};
\begin{scope}[xshift=50,yshift=20]
\html
\end{scope}
\draw [->] (1.5,-0.2) -- (2,0.7);
\node[fancytitle, right=10pt] at (stepone.north west) {网络流量信息采集};

\node[mybox, right=10pt] at (stepone.east) (steptwo){
	\begin{minipage}{0.40\textwidth}
		\tiny
        \begin{flushright}
		\begin{tabular}{@{} c @{}|@{} c @{}|@{} c @{}|@{} c @{}|@{} c @{}|@{} c @{}}
			\hline
			User ID & Request URL & Item \\
			\hline
			1 & www.$\ldots$/bbb & cat \\
			\hline
			2 & www.$\ldots$/aab & car \\
			\hline
			3 & www.$\ldots$/cdd & pig \\
			\hline
		\end{tabular}
		\end{flushright}
	\end{minipage}
};
\begin{scope}[xshift=85,yshift=-25]
\html
\end{scope}
\node[fancytitle, right=10pt] at (steptwo.north west) {语义分析及内容分类};

\draw [->] (3.85,-0.1) -- (4.5,0.1);
\draw [->] (3.75,-0.4) -- (4.5,-0.4);
\draw [->] (3.65,-0.7) -- (4.5,-1);
\node [draw] at (4.8,0.1) {\footnotesize cat};
\node [draw] at (4.8,-0.4) {\footnotesize car};
\node [draw] at (4.8,-1) {\footnotesize pig};

\node[mybox, below=10pt] at (stepone.south) (stepfour){
	\begin{minipage}{0.40\textwidth}
		\tiny
        \begin{center}
		\begin{tabular}{@{} c @{}|@{} c @{}|@{} c @{}|@{} c @{}|@{} c @{}|@{} c @{}}
			\hline
			 & Item 1 & Item 2 & Item 3 \\
			\hline
			User 1 & \textbf{0.7} & 0.8 & 0.9 \\
			\hline
			User 2 & 0.1 & 0.6 & \textbf{0.5} \\
			\hline
			User 3 & 0.2 & \textbf{0.4} & 0.3 \\
			\hline
		\end{tabular}
		\end{center}
	\end{minipage}	
};

\node[mybox, below=10pt] at (steptwo.south) (stepthree){
	\begin{minipage}{0.40\textwidth}
		\tiny
        \begin{center}
		\begin{tabular}{@{} c @{}|@{} c @{}|@{} c @{}|@{} c @{}|@{} c @{}|@{} c @{}}
			\hline
			 & Item 1 & Item 2 & Item 3 \\
			\hline
			User 1 & ? & 0.8 & 0.9 \\
			\hline
			User 2 & 0.1 & 0.6 & ? \\
			\hline
			User 3 & 0.2 & ? & 0.3 \\
			\hline
		\end{tabular}
		\end{center}
	\end{minipage}
};
\node[fancytitle, right=10pt] at (stepfour.north west) {协同过滤预测};
\node[fancytitle, right=10pt] at (stepthree.north west) {数据格式预处理};

\draw[vecArrow] (stepone) to (steptwo);
\draw[vecArrow] (steptwo) to (stepthree);
\draw[vecArrow] (stepthree) to (stepfour);
\end{tikzpicture}
\end{center}
\end{frame}
%%%%%%%%%%%%%%%%%%%%%%


\section{主要模块介绍}
\subsection{用户数据采集及处理}
%%%%%%%%%%%%%%%%%%%%%%
\begin{frame}
	\frametitle{主要模块介绍}
		\begin{block}{用户数据采集及处理}
		本模块主要工作是从网络运营商路由器或者交换机流量记录,以及大型互联网公司服务器访问记录中采集用户行为信息,并按以下格式保存在分布式系统中。
		\begin{center}
		\begin{tabular}{c|c|c|c|c|c}
	\hline
	User ID & Request URL & Time & Location & Content & Device \\
	\hline
\end{tabular}
\end{center}
其中Content代表HTML等网络文件。
		\end{block}
\end{frame}
%%%%%%%%%%%%%%%%%%%%%%

\subsection{数据预处理}
%%%%%%%%%%%%%%%%%%%%%%
\begin{frame}
	\frametitle{主要模块介绍}
		\begin{block}{数据预处理}
		本模块主要工作是对上一模块收集到数据中HTML等网络文件进行语义分析。语义分析得出的关键词与已有的关键词列表对比,更新关键词列表,同时生成用户兴趣信息:
		\begin{center}
\begin{tabular}{c|c|c|c|c|c}
	\hline
	User ID & Item & Time & Request URL & Location & Device \\
	\hline
\end{tabular}
\end{center}
		再将以上格式的信息保存于分布式系统,并初步构建User-Item矩阵。
		\end{block}
\end{frame}
%%%%%%%%%%%%%%%%%%%%%%

\subsection{基于用户历史的协同过滤算法}
%%%%%%%%%%%%%%%%%%%%%%
\begin{frame}
	\frametitle{主要模块介绍}
		\begin{block}{基于用户历史的协同过滤算法}
		计算用户$x$与用户$y$之间的相似程度:
\begin{equation*}
S_{x,y} = \frac{\sum_{i\in I_{xy}}(r_{x,i}-\bar{r}_x)(r_{y,i}-\bar{r}_y)}{\sqrt{\sum_{i\in I_{xy}}(r_{x,i}-\bar{r}_x)^2}\sqrt{\sum_{i\in I_{xy}}(r_{y,i}-\bar{r}_y)^2}}
\end{equation*}
计算用户$u$对物品$i$评分的预测值$r_{u,i}$:
\begin{equation*}
r_{u,i} = \bar{r}_u + \frac{\sum_{u'\in U^k}(r_{u',i}-\bar{r}_{u'})S_{u,u'}}{\sum_{u'\in U^k}S_{u,u'}}
\end{equation*}
		\end{block}
\end{frame}
%%%%%%%%%%%%%%%%%%%%%%


\subsection{基于模型的协同过滤算法}
%%%%%%%%%%%%%%%%%%%%%%
\begin{frame}
	\frametitle{主要模块介绍}
		\begin{block}{基于模型的协同过滤算法}
		$R$为$|C|\times|P|$的User-Item矩阵,其中包含了用户与物品间交互的历史信息。
		首先把$R$渐近地分解为两个秩为$r$的特征矩阵$U$和$M$,使得$R\approx UM$。
$|C|\times r$的矩阵$U$代表用户集的潜在因子,即$R$的行向量。
$r\times |P|$的矩阵$M$代表物品集的潜在因子,即$R$的列向量。
而用户和物品之间的交互强度的预测值可以在低维的矩阵空间中进行,即$u_i^\tau m_j$,其中$u_i$是代表用户$i$的向量,$m_j$是代表物品$j$的向量。
		\end{block}
\end{frame}
%%%%%%%%%%%%%%%%%%%%%%

\section{真实数据测试及性能分析}
\subsection{测试环境}
%%%%%%%%%%%%%%%%%%%%%%
\begin{frame}
	\frametitle{真实数据测试及性能分析}
		\begin{block}{测试环境}
		测试环境包括4台安装了64位Ubuntu14.04以及Hadoop2.6.4的服务器,每台服务器配置为:
		\begin{itemize}
		\item 包含2个Intel的Xeon的CPU,型号为E5-2630,核心频率2.30GHz;
		\item 2GB内存;
		\item 40GB硬盘空间。
	\end{itemize}
		其中一台服务器作为Master,另外三台服务器作为Slave。
		\end{block}
\end{frame}
%%%%%%%%%%%%%%%%%%%%%%

\subsection{测试数据}
%%%%%%%%%%%%%%%%%%%%%%
\begin{frame}
	\frametitle{真实数据测试及性能分析}
		\begin{block}{测试数据}
		由于测试数据的限制,本文无法对系统整体测试,只能按照各模块的功能进行测试。
		在协同过滤预测步骤中,采用了4个不同的数据集,分别是:
		\begin{itemize}
		\item MovieLens的144MB的数据集,其中包含用户对电影的评分数据;
		\item 2015年阿里移动推荐算法比赛的129MB的数据集,其中包含淘宝用户与线上商品的交互信息;
		\item 雅虎音乐1.5GB的C15和R2数据集,其中包含用户对音乐、歌手、专辑的评分。
	\end{itemize}
	\end{block}
\end{frame}
%%%%%%%%%%%%%%%%%%%%%%

\subsection{基于历史的协同过滤算法测试结果}
%%%%%%%%%%%%%%%%%%%%%%
\begin{frame}[shrink]
	\frametitle{真实数据测试及性能分析}
		\begin{block}{基于历史的协同过滤算法测试结果}
		\begin{center}
\begin{tabular}{c|c|c}
 \hline
数据集 & 训练集大小 & F1 score \\ \hline
MovieLens & 576MB & 0.0823 \\ \hline
Aliyun contest & 516MB & 0.0644 \\ \hline
Yahoo! R2 & 6.4GB & 0.0876 \\ \hline
Yahoo! C15 & 6.0GB & 0.0713 \\ \hline
\end{tabular}
\end{center}
\begin{equation*}
precision= \frac{|PredictionSet\cap ReferenceSet|}{|PredictionSet|}
\end{equation*}
\begin{equation*}
recall= \frac{|PredictionSet\cap ReferenceSet|}{|ReferenceSet|}
\end{equation*}
\begin{equation*}
F= 2* \frac{precision*recall}{precision+recall}
\end{equation*}
	\end{block}
\end{frame}
%%%%%%%%%%%%%%%%%%%%%%

\subsection{基于模型的协同过滤算法测试结果}
%%%%%%%%%%%%%%%%%%%%%%
\begin{frame}[shrink]
	\frametitle{真实数据测试及性能分析}
		\begin{block}{基于模型的协同过滤算法测试结果}
		\pgfplotsset{compat=1.13}
\begin{center}
\begin{tikzpicture}
\begin{axis}[
    ybar,
    enlargelimits=0.15,
    legend style={at={(0.5,-0.2)},
      anchor=north,legend columns=-1},
    ylabel={每个map阶段平均运行时间(s)},
    xlabel={特征空间维数$r$},
    symbolic x coords={U10,M10,U20,M20,U50,M50,U100,M100},
    xtick=data,
    nodes near coords,
    nodes near coords align={vertical},
    x tick label style={rotate=45,anchor=east},
    ]
    \addplot coordinates {(U10,42) (M10,43) (U20,41) (M20,42)
         (U50,35)(M50,48) (U100,37)(M100,60)};
    \addplot coordinates {(U10,46) (M10,52) (U20,44) (M20,59)
         (U50,52)(M50,96) (U100,87)(M100,161)};
	 \legend{MovieLens,Yahoo! C15}
\end{axis}
\end{tikzpicture}
\end{center}
	\end{block}
\end{frame}
%%%%%%%%%%%%%%%%%%%%%%

\subsection{Hadoop节点并行性能测试结果}
%%%%%%%%%%%%%%%%%%%%%%
\begin{frame}[shrink]
	\frametitle{真实数据测试及性能分析}
		\begin{block}{Hadoop节点并行性能测试结果}
		\begin{center}
\pgfplotsset{compat=1.13}
\begin{tikzpicture} 
\begin{axis}[
	title={$S_p=T_1/T_p$},
    ylabel={$S_p$},
    xlabel={计算节点数},
    grid=major,
    legend style={
    	font=\tiny,
	    cells={anchor=west},
        legend pos=north west,
    },
    xmin=0,
    ymin=0,
]
\addplot table {data_d1.dat};
\addplot table {data_d2.dat};
\addplot table {data_d3.dat};
\addplot table {data_d4.dat};
\legend{10KB datasize,
100KB datasize,
1MB datasize,
10MB datasize,
}
\end{axis}
\end{tikzpicture}
\end{center}
	\end{block}
\end{frame}
%%%%%%%%%%%%%%%%%%%%%%
\end{document}