\chapter{用户上网数据采集与处理}
本章主要介绍用户上网如何留下信息,如何将上网信息转变为推荐系统所需要的训练集。其中包括了以下内容:网络流量信息采集、打开URL后发生了什么、如何给原数据库中没有的url分类、如何格式化、存储
\section{网络信息传递流程}
互联网用户通过各类软件发送、接受信息。对于不同的软件,有不同的信息发送、接受格式,称作协议。以大多数网络浏览器使用的基本HTTP协议来说,其请求的头部有以下常用的信息 \\
\begin{center}
\tablecaption{HTTP协议请求头部常用的信息}
\begin{tabular}{l|p{10cm}}
 \hline
头部名称 & 描述 \\ \hline
Accept & Content-Types that are acceptable for the response \\ \hline
Content-Type & The MIME type of the body of the request (used with POST and PUT requests) \\ \hline
Date & The date and time that the message was sent \\ \hline
Host & The domain name of the server (for virtual hosting), and the TCP port number on which the server is listening. The port number may be omitted if the port is the standard port for the service requested \\ \hline
Referer & This is the address of the previous web page from which a link to the currently requested page was followed \\ \hline
User-Agent & The user agent string of the user agent \\
\hline
\end{tabular}
\end{center}
根据以上HTTP请求头部我们可以获得用户请求网站URL,请求时间,使用浏览器类型信息。
这里画张图给出请求和反馈流程
当然对于纪录了所有网络流量包的情况,我们甚至可以解压信息包,获得网页信息。对于使用其他协议的软件来说,只要按照该协议的约定,也能正确解压信息包,获取其中内容。
\section{网络流量信息采集与处理}
目前,国内外比较流行使用的网络流量信息采集方式有Cisco公司开发的NetFlow工具。支持NetFlow的路由器和交换机能够收集所有接口上IP交换数据,并且能把数据导出到NetFlow信息分析服务器上。
在“基于Hadoop的网络流量分析系统的研究与应用.pdf”的[17]-[22],中介绍了一些分布式网络流量分析系统实现。
在“基于流量监测的网络用户行为分析.pdf”的P22页有插图,
基于网络流量监测的移动互联网特征研究.pdfP28页有介绍。

本文根据后面系统模块所需信息设计采集模块以及处理集群。对于每个地区的路由器或者交换机,都连接到信息分析服务器。信息分析服务器纪录如下信息,将不是Content-Type不是内容的文件过滤,比如js和css文件。
\begin{enumerate}
 \item User ID
 \item Request URL
 \item Time
 \item Location
 \item Content
 \item Device
\end{enumerate}
从URL能读取出部分信息,有时也包含关键词。我们主要分析HTML文件。HTML文件由各种标签包含各种文字信息。首先要去除各种标签,在Boilerplate Detection Using Shallow Text Features中介绍如何监督学习去除标签。在Mining the Social web 2nd P183
将语义分析网页,根据内容取主要标记。
然后会有特征向量重合以及极度稀疏化的情况,怎么处理
以上能hadoop吗?
之后将生成的记录传入下一个模块
\begin{enumerate}
 \item User ID
 \item Request URL
 \item Time
 \item Location
 \item Category
 \item Key
 \item Device
\end{enumerate}

\section{基于大数据处理的推荐算法}
前一章简单描述了经典推荐算法,现在将其推广至包含超出用户兴趣矩阵之外信息的推荐算法,并且基于Hadoop上实现推荐算法。