% -*- coding: utf-8 -*-


\begin{zhaiyao}
随着互联网的高速发展,用户留下越来越多的网络行为信息。基于这些用户信息,分析用户行为和研究用户模型,研发适合海量数据的自然语义分析、用户兴趣爱好挖掘等算法对预测用户行为及商品推荐至关重要。
本文基于Hadoop和MapReduce大数据处理平台,针对大数据精准营销的难点,实现分布式协同过滤推荐系统并构建用户兴趣矩阵,开展数据挖掘技术、场景化实践的实证研究,构建面向新媒体业务的场景化智能营销平台,并针对新媒体公司自有海量数据的特征与需求进行系统优化,提高资源利用率及推荐效果,从而全面提升公司的用户运营和内容运营能力。
\end{zhaiyao}

\begin{guanjianci}
Hadoop和MapReduce;海量数据处理;自然语义分析;数据挖掘;协同过滤推荐
\end{guanjianci}



\begin{abstract}
Today Internet users, typically 4G mobile users, leave billions tons of data on Servers. Base on such big data, it's urgent to develop algorithm about natural semantic analysis, text mining to predict users' behavior and recommand related content.
This paper focus on big data analysis and natural semantic analysis, develop some data mining algorithm and build a system of distributed collaborative filtering recommendation based on Hadoop and MapReduce, to pop up Internet company's profit.
\end{abstract}



\begin{keywords}
Big data; Natural semantic analysis; Data mining; Hadoop and MapReduce; Collaborative filtering
\end{keywords} 