% 令$A$为$|C|\times |P|$的矩阵,其中包含了所有用户集$|C|$与物品集$|P|$之间的交互数据。
% 即如果用户$i$与物品$j$有交互行为,则$a_{ij}$为一个记录这次交互行为强度的数值。
% 若用户$i$对物品$j$无交互行为,则$a_{ij}=?$。
% 最基本的协同过滤算法所研究的问题为:基于$A$,给每位用户推荐一个不包含该用户已经评价过物品的列表,其中的物品按照符合用户兴趣程度递减排序。

% 我们先预处理每个物品评分向量,记为函数\textbf{preprocess()}
% \[
% \hat{i}=preprocess(i)\qquad \hat{j}=preprocess(j)
% \]
% 对于标准的协同过滤算法,\textbf{preprocess()}函数用cos相似度\parencite{Singhal2001Modern}计算,也可以采用其他方式如Pearson相关系数计算。
% \[
% preprocess(i)=\frac{i-\bar{i}}{\parallel i-\bar{i}\parallel_2}
% \]
% 下一步使用\textbf{norm()}函数处理$\hat{i}$与$\hat{j}$
% \[
% n_i = norm(\hat{i})\qquad n_j = norm(\hat{j})
% \]
% 使用自定义的点乘处理$\hat{i}$与$\hat{j}$
% dot_{ij} = \hat{i}\dot \hat{j}
% 然后,物品相似矩阵$S$计算为
% \begin{eqnarray}
% S_{ij} &=& similarity(dot_{ij},norm(\hat{i}),norm(\hat{j}))
% &=& \hat{i}\dot \hat{j}
% \end{eqnarray}

% \begin{center}
% \tablecaption{相似度计算方式}
% \begin{tabular}{c|c|c|c}
% 	\hline
% 	方式 & preprocess & norm & similarity \\
% 	\hline
% 	Cosine & $\frac{\nu}{\parallel \nu \parallel_2}$ & - & $dot_{ij}$ \\
% 	\hline
% 	Pearson correlation & $\frac{\nu-\bar{\nu}}{\parallel \nu-\bar{\nu}\parallel_2}$ & - & $dot_{ij}$ \\
% 	\hline
% 	Euclidan distance & - & $\hat{\nu}^2$ & - & $\sqrt{n_i-2\dot dot_{ij} +n_j}$
% \end{tabular}
% \end{center}

% 最后用户$u$对物品$i$的预测评分$r_{ui}$可以计算为
% \[
% r_{ui} = \mu + \frac{\sum_{j\in S^k(i,u)} S_{ij}(A_{uj}-\mu)}{\sum_{j\in S^k(i,u)} S_{ij}}
% \]
% 其中$\mu$为平均评分,$S^k(i,u)$代表与用户$u$距离最近的$k$个用户。

% \section{低阶矩阵分解}
% 在推荐系统中通常有超大的User-Item矩阵,即矩阵有超高维度,造成计算极其复杂。
% 由于许多维度是重合的,于是我们可以寻找一些隐藏的维度来表示这些重合的维度,该方法称为潜在因素模型。
% 以下描述经典的穿行算法。
% 对于给定$|C| \times |P|$的秩为$r$的User-Item矩阵$A$,要找到$|C| \times r$的矩阵$U$(代表用户的潜在特征)和$r \times |P|$的矩阵$M$(代表物品的潜在特征),满足
% \[
% A \equiv U \cdot M^\tau
% \]
% 其中矩阵$U$的每一行代表一个用户为$u_i$,矩阵$M$的每一行代表一个物品为$m_j$。目标是最小化
% \[
% argmin \sum_{i,j}(a_{i,j}-u_i\cdot m_j)^2
% \]
% 给定一个阈值$\epsilon$,通过以下步骤找到矩阵$U$和$V$:
% \begin{enumerate}
% 	\item 随机选择一个矩阵$M$;
% 	\item 固定矩阵$M$,优化矩阵$U$;对逐个用户$u_i$进行优化,即固定$m_j$,需要找到
% 	\[
% 	argmin_{u_i} \sum_{j\in P_i}(p_{i,j}-u_i\cdot m_j)^2
% 	\]
% 	这是线性最小二乘方程,可以得到
% 	\[
% 	u_i=(V_{*,i}^{\tau}V_{*,i})^{-1}V_{*,i}P_{*i}
% 	\]
% 	其中$V_{*,i}$是有来自用户$u_i$的评价的$V$的子集。
% 	\item 固定矩阵$U$,优化矩阵$V$,与上一步相似;
% 	\item 循环第2、3步,直到相关系数的变化量小于$\epsilon$,即可得到收敛解。
% \end{enumerate}